\documentclass[a4paper]{article}
\usepackage[a4paper]{geometry}
\usepackage{amsmath}
\usepackage{amssymb}
\usepackage[utf8]{inputenc}
\usepackage{graphicx}
\usepackage{booktabs}
\usepackage[russian]{babel}
\usepackage{listings}


\usepackage{color} %% это для отображения цвета в коде
\usepackage{listings} %% собственно, это и есть пакет listings

\definecolor{gry}{rgb}{0.6,0.6,0.6}
\usepackage{caption}
\DeclareCaptionFont{white}{\color{white}} %% это сделает текст заголовка белым
%% код ниже нарисует серую рамочку вокруг заголовка кода.
\DeclareCaptionFormat{listing}{\colorbox{gry}{\parbox{\textwidth}{#1#2#3}}}
\captionsetup[lstlisting]{format=listing,labelfont+=white,labelfont+=bf,textfont+=bf,textfont+=white}

\lstset{ %
language=C,                 % выбор языка для подсветки (здесь это С)
basicstyle=\small\sffamily, % размер и начертание шрифта для подсветки кода
numbers=left,               % где поставить нумерацию строк (слева\справа)
numberstyle=\tiny,           % размер шрифта для номеров строк
stepnumber=1,                   % размер шага между двумя номерами строк
numbersep=5pt,                % как далеко отстоят номера строк от подсвечиваемого кода
backgroundcolor=\color{white}, % цвет фона подсветки - используем \usepackage{color}
showspaces=false,            % показывать или нет пробелы специальными отступами
showstringspaces=false,      % показывать или нет пробелы в строках
showtabs=false,             % показывать или нет табуляцию в строках
frame=single,              % рисовать рамку вокруг кода
tabsize=2,                 % размер табуляции по умолчанию равен 2 пробелам
captionpos=t,              % позиция заголовка вверху [t] или внизу [b] 
breaklines=true,           % автоматически переносить строки (да\нет)
breakatwhitespace=false, % переносить строки только если есть пробел
escapeinside={\%*}{*)}   % если нужно добавить комментарии в коде
}
\usepackage[T2A]{fontenc}
\renewcommand{\lstlistingname}{Листинг}

\title{Десять применений define-ов в языке Си}
\date{20 февраля 2017 г.}
\author{Микоян Филипп}

\begin{document}		\pagenumbering{arabic}
	\maketitle

	\section{Краткое введение в макросы}
	
	\subsection{Терминология}
	
	\begin{description}
	\item[•] \textbf{Препроцессор} - программа, обрабатывающая исходные файлы перед компиляцией. 
	\item[•] \textbf{Директива} - указание. 
	\item[•] \textbf{Макрос} - символьное имя, заменяемое препроцессором на последовательность символов, возможно зависящую от аргументов макроса. Макросы могут применяться для сокращения записи повторяющихся/мало отличающихся последовательностей инструкций.
	\item[•] $\textbf{\#define}$ - директива препроцессору создать макрос, т.е. заменить все последующие употребления имени макроса на определённую последовательность символов.
	\end{description}
	
	\subsection{Синтаксис}
	
	\begin{description}
	\item[•]\textbf{Макрос без аргументов} \\
			$\#define\ identifier\ token-stringopt$ \\
				\begin{lstlisting}[caption={Пример применения макроса без аргументов}]
#include <stdio.h>

#define mew int

int main(void)
{
    mew a = 5;
    printf("%i", a/2); // 2 is to be printed
    return 0;
}
				\end{lstlisting}
			Этот пример, вероятно, не совсем безопасного применения макроса иллюстрирует, как можно 'переименовать' тип целочисленной переменной.
			
\newpage			
			
	\item[•]\textbf{Макрос с аргументами} \\
		$\#define\ identifier\ (identifieropt1,\ ...,\ identifieroptn)\ token\_sequence$
				\begin{lstlisting}[caption={Пример применения макроса с аргументами}]
#include <stdio.h>

#define sqr(x) ((x)*(x)) 

int main(void)
{
    int a = 5;
    printf("%i", sqr(a)); //25 is to be printed
    return 0;
}
				\end{lstlisting}
				В данном примере каждый \textit{'x'} из выражения $((x)*(x))$ непосредственно заменяется на то, что подаётся в макрос в качестве аргумента, будь то строка, выражение, символ, идентификатор, число и т.д. 
	\end{description}
	
	\section{$\bf{Macros \approx Evil}$}
	
	Макросы следует использовать с осторожностью, придерживаясь следующих правил:
	\begin{description}
	\item[•] В названиях макросов употреблять только заглавные буквы, использовать префиксы, содержащие имя проекта и/или названия, потенциально редко применяемые остальными программистами
	\item[•] Использовать макросы только если это оправдано, всегда задаваться вопросом о возможности замены макроса функцией
	\item[•] Описывать макросы с простой реализацией
	\item[•] Описывать только те макросы, которые не требуют отладки
	\item[•] Избегать макросов, неявно что-либо изменяющих
	\item[•] В случае необходимости описания больших макросов/макросов со сложной реализацией, обязательно снабжать их комментариями
	\end{description}
	В случае соблюдения вышеописанных правил, единственным серьёзным недостатком макросов можно назвать разве что "разбухание" исходного файла, обработанного препроцессором, и, как следствие, более длительную компиляцию и увеличение размера скомпилированной программы. 
	
\newpage	
	
	\section{Допустимые оправдания использования макросов}
	
	Макросы обладают несколькими достоинствами, которые, по моему мнению, могут являться единственным оправданием их использования. К ним относятся следующие особенности макросов(перечислены в порядке убывания важности):
	\begin{description}
	\item[•] Позволяют избежать копипаста
	\item[•] Подчастую могут увеличить читабельность кода
	\item[•] Позволяют узнать \textit{имя} переменной, переданной в макрос, а не только её значение
	\item[•] Позволяют \textit{слить} два токена в один
	\item[•] Инвариантность вида макроса относительно типов передаваемых в него аргументов (см. п.4.3)
	\item[•] Используются при создании $\#include\ guard$-ов
	\item[•] Вкупе с остальным функционалом препроцессора(в частности, include-ами) позволяет выносить некоторые описания в отдельные файлы, которые можно назвать конфигурационными
	\item[•] Предописанные макросы бывают незаменимы
	\item[•] При грамотном применении создают новый уровень абстракций(в Си позволяет следовать принципам Domain-driven design, имитировать Domain-specific language)
	\end{description}
	
\newpage
	
	\section{Типичные 'use cases'(примеры применения) макросов}
	
	\subsection{'Пустые' макросы}
	
	\subsubsection{'Флажки' в исходном коде}
	Бывает удобно использовать 'пустой' макрос, например, 'Govnocode', чтобы помечать в исходном коде те места, которые следует отрефакторить в первую очередь, и которые, в связи с близостью дедлайна, трудно написать хорошо сразу.
	\begin{lstlisting}[caption={Govnocode}]
#include <stdio.h>

#define Govnocode 

const int CONST1 = 666;
const int CONST2 = 678;

int main(void)
{
    char c = '\0';
    while (!scanf("%c", &c));
    switch (c)
    {               Govnocode //Replace with macros
        case '^':
            {
                printf("%d", CONST1 ^ CONST2);
            }; break;
        case '+':
            {
                printf("%d", CONST1 + CONST2);
            }; break;
        case '-':
            {
                printf("%d", CONST1 - CONST2);
            }; break;
        case '*':
            {
                printf("%d", CONST1 * CONST2);
            }; break;
        case '/':
            {
                printf("%d", CONST1 / CONST2);
            }; break;
        default: return 1;
    }
    return 0;
}
	\end{lstlisting}

\newpage

\subsubsection{$\textbf{\#include\ guards}$}

\paragraph{$\textbf{\#include\ guard}$} - \ конструкция, используемая с целью избежать "двойное подключение" директивой $\#include$.
	\begin{lstlisting}[caption={$\textbf{\#include\ guard}$}]
#ifndef SUPER_HEADER_INCLUDED
#define SUPER_HEARER_INCLUDED

struct SuperStruct
{
    int superValue;
    char superChar;
};

#endif
	\end{lstlisting}
	
\subsubsection{Условная компиляция}

\paragraph{Условная компиляция} - \ совокупность директив препроцессора, позволяющая компилировать или пропускать часть программы в зависимости от выполнения некоторого условия, в частности, от существования того или иного макроса. К директивам условной компиляции относятся $\textbf{\#ifdef},\ \textbf{\#ifndef},\ \textbf{\#else},\ \textbf{\#endif}$
	\begin{lstlisting}[caption={Условная компиляция}]
#include <stdio.h>

//#define TEST_MODE
//#define SILENT_MODE

int main(void)
{
#ifndef SILENT_MODE

#ifdef TEST_MODE
    printf("Test mode");
#else  //TEST_MODE
    printf("Normal mode"); //This is to be printed
#endif //TEST_MODE

#endif //SILENT_MODE
    return 0;
}	
	\end{lstlisting}
	Следует отметить, что т.к. директивы условной компиляции обычно пишут без ведущих пробелов, т.е. без отступов, следует снабжать каждую директиву $\textbf{\#else},\ \textbf{\#endif}$ комментарием, указывающим, какой блок условной компиляции даннная директива закрывает. 
	
\newpage

\subsection{Именование констант}

	Зачастую начинающие программисты используют макросы для именования констант, к примеру:
	\begin{lstlisting}[caption={Именованная константа}]
#define MAX_STRING_LENGTH 256
	\end{lstlisting}
	Эта практика небезопасна, т.к. вместо числа '256' программист может случайно написать, к примеру, вызов функции, о чём впоследствии забыть. Это может привести к долгой, изнурительной и безрезультатной отладке, а потому вместо макросов в данном случае рекомендуется использовать enum. В случае, если несколько констант имеет смысл объединить в одну группу, следует поместить их в один enum.
	\begin{lstlisting}[caption={Enums}]
enum { MAX_STRING_LENGTH = 256 }; //Single constant

enum STKERRORS //Block of constants
{
	STKERR_NULLPTR,
	STKERR_NULLDATAPTR,
	STKERR_NEGSIZE,
};
	\end{lstlisting}
	
\subsection{Макросы как функции}

	Вообще говоря, макросы всегда нужно стараться заменить функциями, т.к. их проще отлаживать и они более безопасны.
	Единственным обоснованием применения макросов вместо функций может являться то, что один и тот же макрос может принимать аргументы разных типов.
\newpage
	\begin{lstlisting}[caption={Macro Function}]
#include <stdio.h>

//#define DUMB_SQR

#ifdef DUMB_SQR
double sqrD(double a)
{
    return a * a;
};

int sqrI(int a)
{
    return a * a;
};

char sqrC(char a)
{
    return a * a;
};
#else
 
#define sqr(a) ((a)*(a))

#endif

int main(void)
{
    printf("%i", sqr(5));
    return 0;
}
	\end{lstlisting}
	В данном подпункте следует особо отметить, что в макросах нужно ставить как можно больше скобок, учитывать все возможные значения аргументов. Это проиллюстрировано в следующем примере:
		\begin{lstlisting}[caption={Wrong mul}]
#include <stdio.h>

#define mul(a, b) a * b //Wrong 'mul' macro - not enough brackets.

const int CONST = 666;

int main(void)
{
    printf("%i", mul(CONST + 5, 3)); //681 is to be printed, 2013 expected
}
	\end{lstlisting}
	
\newpage	
	
	\subsection{Макросы для получения имени переменной}
	\begin{lstlisting}[caption={Получение имени переменной макросов}]
#include <stdio.h>
#include <assert.h>
#include <stdlib.h>
#include <string.h>

enum { MAX_STRING_LEN = 256 };

struct Point
{
    double x;
    double y;
};

void pointDump_(struct Point* point, const char pointName[MAX_STRING_LEN])
{
    assert(point);

    printf("struct Point '%s' [%p]\n", pointName, point);
    printf("{\n"
           "\tx = %lg\n"
           "\ty = %lg\n"
           "}\n", point->x, point->y);
}
#define pointDump(pnt)                          \
{                                               \
    assert(strlen(#pnt) <= MAX_STRING_LENGTH);  \
    pointDump_(pnt, #pnt);                      \
}

int main(void)
{
    struct Point* mySuperPoint = (struct Point*)calloc(1, sizeof(*mySuperPoint));
    pointDump(mySuperPoint);
    return 0;
}
	\end{lstlisting} 
	В данном примере ($\#pnt$) - строковое представление переданного в макрос параметра. \\
	\textbf{Вывод программы Point dump.c:} \\ \\
\begin{tabular}{|l}
struct Point 'mySuperPoint' [0x55fa8e58b010] \\
\{ \\
\ \ \ \ x = 0\\
\ \ \ \ y = 0\\
\}
\end{tabular} \\
Обратите внимание на обратные слэши в конце каждой строки в описании макроса. Они используются для описания многострочных макросов. Их нужно ставить в конце строки(обратите внимание, что после них не должны стоять даже пробелы), которую необходимо перенести.

\newpage

\subsection{Макросы для склейки токенов}
	\begin{lstlisting}[caption={Склейка токенов макросами}]
int compare_chicks(const void *chick1, const void *chick2)
{
    #define boobs(i)   (*((const int **)chick##i)) [0]
    #define waist(i)   (*((const int **)chick##i)) [1]
    #define hippies(i) (*((const int **)chick##i)) [2]
    #define sum(i)     (boobs(i) + waist(i) + hippies(i))
    
    int sum1 = sum(1);
    int sum2 = sum(2);
    
    int firstChickIsBest = (sum1 < sum2);

    if (sum1 == sum2)
    {
        firstChickIsBest = boobs(1) > boobs(2);
        if (boobs(1) == boobs(2))
        {
            firstChickIsBest = waist(1) < waist(2);
            if (waist(1) == waist(2))
            {
                firstChickIsBest = hippies(1) > hippies(2);
            }
        }
    }
    
    return !firstChickIsBest;
    
    #undef sum
    #undef boobs
    #undef waist
    #undef hippies
}
	\end{lstlisting} 
	В данном примере особое внимание следует обратить на двойные решётки('$\#\#$') в описаниях макросов. Если бы их не было, препроцессор воспринял бы последовательность символов \textit{chicki} как идентификатор и напрямую заменял бы, скажем, \textit{waist(0)} на \textit{(*((const int **)chicki)) [1]}, т.е. это выражение даже не зависело бы от передаваемого параметра. \\
	С помощью двойной решётки мы указываем компилятору, что эти две последовательности символов(\textit{chick} и \textit{i}) следует сначала проверить на необходимость замены на аргументы, а затем 'склеить'.\\
	Также следует отметить присутствие директив препроцессору $\#undef$. Эти директивы используются для ограничения области видимости макросов. Их обязательно следует ставить там, где кончается область видимости макроса. \\
	Отмечу также, что чем шире область видимости макросов, тем 'привередливей' следует относиться к их названиям, и наоборот.
	
\newpage

\subsection{Макросы для замены однотипных case-ов и функций}

	\begin{lstlisting}[caption={Switch and macros}]
#include <stdio.h>

const int CONST1 = 666;
const int CONST2 = 678;

int main(void)
{
    char c = '\0';
    while (!scanf("%c", &c));
    
#define caseOperator(operChar, oper)      \
case operChar:                            \
    {                                     \
        printf("%d", CONST1 oper CONST2); \
    }; break;

    switch (c)
    {
  	    caseOperator('^', ^);
  	    caseOperator('+', +);
  	    caseOperator('-', -);
  	    caseOperator('*', *);
        caseOperator('/', /);
        default: return 1;
    }
    
#undef caseOperator
    return 0;
}
	\end{lstlisting}
Этот пример представляет собой переделанный Листинг 3 из пункта 4.1.1. \\ 

В данном листинге хорошо проиллюстрировано, как можно использовать макросы для устранения копипаста - переписывания/копирования участка кода в разные части программы. Метод состоит в следующем: создаётся макрос, целиком содержащий в себе этот участок кода. Если этот участок в зависимости от той части программы, в которую его нужно вставить, слегка изменяется, то эти изменения производятся путём введения аргументов макроса, отличающихся в разных частях программы.

\newpage

	\begin{lstlisting}[caption={Functions and macros}]
#include <stdio.h>

#define sqrFunc(varType)          \
varType sqr_##varType(varType a); \
                                  \
varType sqr_##varType(varType a)  \
{                                 \
    return a * a;                 \
};                                

sqrFunc(double)
sqrFunc(int)
sqrFunc(char)

#undef sqrFunc

int main(void)
{
    printf("%i", sqr_int(5));
    return 0;
}
	\end{lstlisting} 
Данный пример представляет собой переделанный Листинг 8 из пункта 4.3. \\

Кроме приведённых примеров к данному пункту также можно отнести, например, применение макросов для описания похожих функций, отличающихся лишь наборов передаваемых в них параметров. \\



\subsection{Макросы, описанные в отдельных($\approx$ конфигурационных) файлах}


	\begin{lstlisting}[caption={Файл Enums.h, использующий макросы, описанные в Keywords.h}]
#ifndef ENUMS_H_INCLUDED
#define ENUMS_H_INCLUDED

enum LangKeyword 
{
    #define LANG_KEYWORDS
    
    #define LANG_KEYWORD(keywd, type) LANGKWD_##keywd,
    #include "Keywords.h"
    #undef LANG_KEYWORD
    
    #undef LANG_KEYWORDS
    LANGKWD_KWDQT,
};



enum LangKeywordType
{
    #define LANG_TYPES
    
    #define LANG_TYPE(type) LANGKWT_##type,
    #include "Keywords.h"
    #undef LANG_TYPE
    
    #undef LANG_TYPES
};

#endif
	\end{lstlisting} 
	\begin{lstlisting}[caption={Файл, использующий макросы, описанные в Keywords.h}]
#ifdef LANG_MATH_OPERATORS 
//  LANG_MATH_OP( name, char, priority )
//+-------------------------------------------+
    LANG_MATH_OP( ln    , 'l', 0            )
//~~~~~~~~~~~~~~~~~~~~~~~~~~~~~~~~~~~~~~~~~~~~
//--------------------------------------------
//~~~~~~~~~~~~~~~~~~~~~~~~~~~~~~~~~~~~~~~~~~~~
    LANG_MATH_OP( minus , '-', 3            )
//+-------------------------------------------+
#endif

#ifdef LANG_KEYWORDS
//+-------------------------------------------+
//  LANG_KEYWORD( keyword,  keyword_type     )
//~~~~~~~~~~~~~~~~~~~~~~~~~~~~~~~~~~~~~~~~~~~~
//--------------------------------------------
//~~~~~~~~~~~~~~~~~~~~~~~~~~~~~~~~~~~~~~~~~~~~
    LANG_KEYWORD( while,    loop             )
//+-------------------------------------------+ 
#endif

#ifdef LANG_TYPES
//+-------------------------------------------+
//  LANG_TYPE( type             )
//~~~~~~~~~~~~~~~~~~~~~~~~~~~~~~~~~~~~~~~~~~~~
//--------------------------------------------
//~~~~~~~~~~~~~~~~~~~~~~~~~~~~~~~~~~~~~~~~~~~~
    LANG_TYPE( function         )
//+-------------------------------------------+
#endif
\end{lstlisting} 
\subsection{Getter и setter, реализованные с помощью макросов}
	 
\newpage	 
	 
\section{Опция -E}

	Отдельный заголовок хотелось бы выделить, чтобы обратить внимание читателя на опцию компилятора '-E', которая позволяет увидеть исходные файлы такими, какими они стали после обработки препроцессором, т.е. со всеми 'раскрытыми' '$\#define$-ами и $\#include$-ами. \\

	Начинающему пользователю макросов настоятельно рекомендуется воспользоваться данной опцией и изучить возвращённые препроцессором файлы. Также бывает очень полезно воспользоваться данной опцией при неожиданном и необъяснимом поведении программы, в которой, возможно, построена чересчур сложная система макросов.
	
\end{document}